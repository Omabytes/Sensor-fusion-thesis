\let\textcircled=\pgftextcircled
\chapter{Further Work}
\label{chap:furtherwork}

\initial{T}his chapter suggests further work that could be done to improve and build upon the project as it currently stands.

\section{Integration and Real World testing}
The filter code needs to be integrated with the code handling sensory apparatus for input, and then output to the flight controller. As the target code is written largely in the same language this should be a fairly simple process. Following that, further testing of the filter needs to be done once the drone itself is fully functional to finish characterisation of all the noise parameters.

\section{Improved Sensory Apparatus}
Discussed in chapter \ref{chap:sensors} was the promising but infeasible LiDAR system. There could be area here to develop a more cost efficient version of this system using cheap off the shelf components which would greatly enhance the information input available.

\section{Kalman Filter Variants}
In chapter \ref{chap:introkalmanfilter} the idea of Kalman filter variants was looked at and dismissed as unnecessary. In the event of a future change in instrumentation or use conditions, these variants may become useful. The extended Kalman filter is the standard used in many commercially available products. It does however have some aforementioned drawbacks in terms of being suboptimal, not being completely stable under some starting conditions, and generally being difficult to implement \cite{julier2004unscented}. The much lauded alternative to the EKF is the Sigma-point (unscented) Kalman filter, may well be a better alternative to the EKF for further developments of this project \cite{van2004sigma}.

\section{Multiple Filters}
Pushing the data through multiple filters then combining the estimates could be an interesting experiment to see if more accurate estimates can be obtained.

\section{Dynamic Noise Updates}
As has been stressed previously, the Kalman filter relies heavily on accurate characterisation of the sensor behaviour. This also needs to be done prior to run time and the values cannot be altered later on if the sensor behaviour changes. Reasons for characteristic changes could be alteration in environmental conditions or sensor malfunction. Ideally the sensor noise would be able to be be dynamically updated during filter runtime to reflect estimated accuracy. Eom \textit{et al.} \cite{eom2011improved} looked at using a neural network to estimate measurement noise covariance. This could be investigated further,  or a more naive approach could be looked at which performs slightly worse but with the benefit of greatly reduced complexity and computational load.